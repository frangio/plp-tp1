\section{Grafo}

\begin{verbatim}

module Grafo 
    (Grafo, vacio, nodos, vecinos, agNodo, sacarNodo, agEje, lineal, union, clausura) where

import qualified Data.List as List (delete, union, nub)

-- Invariante: ninguna de las listas (nodos y vecinos de nodos) tienen repetidos
data Grafo a = G [a] (a -> [a])

instance (Show a) => Show (Grafo a) where
  show (G n e) = 
    "[\n" ++ concat (map (\x -> " " ++ show x ++ " -> " ++ show (e x) ++ "\n") n) ++ "]"


-- Ejercicio 1
-- Crea un grafo sin nodos.

vacio :: Grafo a
vacio = G [] (\_ -> [])

-- Ejercicio 2
-- Devuelve los nodos de un grafo.

nodos :: Grafo a -> [a]
nodos (G ns fv) = ns

-- Ejercicio 3
-- Devuelve los vecinos de un nodo en un grafo dado.

vecinos :: Grafo a -> a -> [a]
vecinos (G ns fv) = fv

-- Ejercicio 4
-- Devuelve un grafo con un nodo agregado, sin vecinos.
-- Si el nodo ya está presente en el grafo, no tendrá vecinos en nuevo grafo.

agNodo :: Eq a => a -> Grafo a -> Grafo a
agNodo n (G ns fv) = G ns' fv'
        where ns' = List.union [n] ns
              fv' e | e == n    = []
                    | otherwise = fv e


-- Ejercicio 5
-- Devuelve un grafo sin un nodo dado.

sacarNodo :: Eq a => a -> Grafo a -> Grafo a
sacarNodo n (G ns fv) = G ns' fv'
        where ns' = List.delete n ns
              fv' e | e == n    = []
                    | otherwise = List.delete n $ fv e






-- Ejercicio 6
-- Devuelve un grafo con un eje agregado.

agEje :: Eq a =>  (a,a) -> Grafo a -> Grafo a
agEje (n1, n2) (G ns fv) = G ns fv' 
        where fv' e | e == n1   = n2 : fv e
                    | otherwise = fv e


-- Ejercicio 7
-- Devuelve un grafo donde los nodos son todos los de la lista pasada por
-- argumento, y cada nodo tiene como unico vecino al elemento que lo sigue en
-- dicha lista.

lineal :: Eq a => [a] -> Grafo a
lineal ns = G ns fv 
        where fv e = take 1 $ drop 1 $ dropWhile (/= e) ns


-- Ejercicio 8
-- Devuelve la unión de dos grafos.

union :: Eq a => Grafo a -> Grafo a -> Grafo a
union (G ns1 fv1) (G ns2 fv2) = G ns' fv'
        where ns' = List.union ns1 ns2
              fv' e = List.union (fv1 e) (fv2 e)

-- Ejercicio 9
-- Devuelve la clausura reflexo transitiva de un grafo.

clausura :: (Eq a) => Grafo a -> Grafo a
clausura (G ns fv) = G ns fv'
        where fv' e = puntofijo extenderConVecinos [e]
              extenderConVecinos xs = foldl List.union xs (map fv xs)

-- Punto fijo
-- Devuelve f(f(...f(x)...)) = f^n(x), tal que f(f^n(x)) = f^n(x), es decir que
-- es un punto fijo de la función.

puntofijo :: (Eq a) => (a -> a) -> (a -> a)
puntofijo f x = head $ dropWhile (\e -> (f e) /= e) $ iterate f x

\end{verbatim}
